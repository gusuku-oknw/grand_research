\documentclass{jarticle}
\usepackage[dvipdfmx]{graphicx}
\usepackage[top=5truemm,bottom=20truemm,left=10truemm,right=10truemm]{geometry}
\usepackage{amsmath,amssymb,mathtools}
\DeclarePairedDelimiter{\norm}{\lVert}{\rVert}
\usepackage{arydshln}
\usepackage{otf}
\usepackage{multirow}
% \usepackage{wasysym}
\usepackage{titlesec}
\usepackage{float} % Hオプションで図表位置を固定するため

\title{秘密画像共有における pHash 検索を可能にする知覚暗号化の設計}
\author{横田・\CID{08521}水研究室 玉城 洵弥}
\date{2025年度 卒業論文審査会}

\setlength{\columnsep}{20pt}

\titleformat{\section}
  {\normalfont\large\bfseries}
  {\thesection}
  {.5em}
  {}

\makeatletter
\renewcommand{\maketitle}{%
  \begin{center}
    \Large\bfseries \@title
  \end{center}
  \begin{flushright}
    {\large\@author}\par%
    \@date\par
  \end{flushright}
  \vspace{-10pt}
  \centering
   \rule{.48\textwidth}{0.5pt}
   $\surd$
   \rule{.48\textwidth}{0.5pt}
}
\makeatother

\begin{document}
\twocolumn[\maketitle]
\linespread{0.92}\selectfont

\section*{概要}
秘密画像共有 (SIS) で分散保存した画像について,平文化せずに pHash による類似画像検索を行う方式を提案する。
シェアのみを持つサーバでも検索できるよう,
\textbf{$k_1$ で pHash の符号だけを開示し,視覚的にはノイズ化した pHash 整合ダミー画像だけを見せる}
三段階開示モデルを設計した。ここで $1<k_1<k_2\le n$ とし,$k_1$ は検索のみ許可,$k_2$ は完全復元の閾値とする。
シェア数に応じて
\textbf{「ノイズ」「pHash 整合ダミー」「原画像」}
の三段階で情報を開示し,
検索精度と秘匿性の両立を図る。
MS COCO(Common Objects in Context)派生データによる評価では,
pHash 検索精度・処理時間ともに平文と同等であることを確認した。

\section{はじめに}
医療画像や監視映像では「検索は必要だが中身は見せられない」
という要求がある。
既存の類似画像検索(CBIR: Content-Based Image Retrieval)は画像や特徴量を平文で保持するため,
漏洩リスクが高い。
pHash は軽量で実用的な知覚特徴量である一方,
符号を平文で扱うと情報漏洩につながる。
また,既存の秘密画像共有(SIS)は
「復元する/しない」の二択であり,
\textbf{復元せずに検索だけを許可する中間状態}を扱えないという課題がある。

プライバシ保護 CBIR では,
CNN 特徴量を秘密分散や MPC により安全計算する方式が提案されている。
しかし,高次元特徴を前提とするため計算コストが高く,
段階的な情報開示は考慮されていない。
一方,pHash は低次元で高速だが,
暗号技術と統合した研究はほとんど存在しない。
特に,
\textbf{pHash 符号のみを保持したまま視覚情報を消去する設計}
は未検討である。

\section{提案法}
本研究では,
pHash の低周波符号構造と二階層 Shamir 秘密分散を組み合わせ,
三段階の情報開示を実現する。

\subsection*{定式化(三段階 SIS)}
閾値は $1<k_1<k_2\le n$ とし,$k_1$ は検索権限,$k_2$ は原画像復元権限を表す。
配布した $n$ 枚のシェア集合を $A$,$r=|A|$ とし,秘密を $S_0,S_1,S_2$ で定義する。
\[
\text{output}(r)=
\begin{cases}
\text{noise} & (r < k_1) \\
\text{pHash 整合ダミー} & (k_1 \le r < k_2) \\
\text{原画像} & (r \ge k_2)
\end{cases}
\]
\noindent ここで $S_0$ は無意味なノイズ画像,$S_1$ は pHash 整合ダミー生成情報,$S_2$ は原画像復元情報を表す。
\noindent 安全性として $r<k_1$ で $I(S_1;A)=0$,かつ $r<k_2$ で $I(S_2;A)=0$ を要求する。
\noindent また $k_1 \le r < k_2$ では pHash による検索のみ可能($\mathrm{pHash}(S_1)=\mathrm{pHash}(I)$)とする。
\begin{figure}[t]
\centering
\includegraphics[width=0.85\linewidth]{output/fig_scene_cropped.png}
\caption{三段階 SIS の概念図($k_1$:検索のみ,$k_2$:完全復元)}
\label{fig:three-stage}
\end{figure}

\subsection*{pHash 整合ダミー}
低周波 DCT 符号のみを一致させ,
高周波成分をランダムノイズで置換することで,
pHash は一致するが視覚情報は失われた画像を生成する。

\subsection*{二階層 Shamir}
$n=5,\ k_1=2,\ k_2=4$ とし,
$k_1$ で検索権限,
$k_2$ で原画像復元を許可する。

\section{実験}
COCO val2017 派生データ(500 クエリ,20 バリアント)を用いて評価した。
その結果,
\begin{itemize}
  \item 上位1件の検索精度は 100\%(平文・ダミー一致)
  \item 上位5件 86.6\%,上位10件 84.82\%(全バリアント)
  \item 検索時間は約 0.5 ms/query と平文と同等
\end{itemize}
であり,
pHash 整合ダミーが検索性能を劣化させないことを確認した。

\section{おわりに}
pHash 符号一致ダミーと二階層秘密分散を用いた
段階的情報開示方式を提案した。
本方式により,
\textbf{画像を見せずに画像を検索する}という要求を
軽量かつ実用的に実現できることを示した。
今後は大規模データへの拡張や,
CNN 特徴とのハイブリッド化を検討する。

\begin{thebibliography}{9}
\bibitem{xia}
Z. Xia et al., ``A privacy-preserving CBIR scheme based on secret sharing,''
\textit{IEEE Access}, 2020.
\end{thebibliography}

\end{document}











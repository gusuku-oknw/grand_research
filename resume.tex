\documentclass{jarticle}
\usepackage[dvipdfmx]{graphicx}
\usepackage[top=5truemm,bottom=20truemm,left=10truemm,right=10truemm]{geometry}
\usepackage{amsmath,amssymb,mathtools}
\DeclarePairedDelimiter{\norm}{\lVert}{\rVert}
\usepackage{arydshln}
\usepackage{otf}
\usepackage{multirow}
% \usepackage{wasysym}
\usepackage{titlesec}
\usepackage{float} % Hオプションで図表位置を固定するため
\usepackage{url}
\graphicspath{{./output/}{./output/phash_masked_sis_eval/}{./output/phash_masked_sis_eval/figures/}{./output/results/}{./output/results/masked_phash_eval_figs/}}

\title{秘密画像共有における pHash 検索を可能にする\\
知覚暗号化の設計}
\author{横田・\CID{08521}水研究室 玉城 洵弥}
\date{2025年度 卒業論文審査会}

\setlength{\columnsep}{20pt}

\titleformat{\section}
  {\normalfont\large\bfseries}
  {\thesection}
  {.5em}
  {}

\makeatletter
\renewcommand{\maketitle}{%
  \begin{center}
    \Large\bfseries \@title
  \end{center}
  \begin{flushright}
    {\large\@author}\par%
    \@date\par
  \end{flushright}
  \vspace{-10pt}
  \centering
   \rule{.48\textwidth}{0.5pt}
   $\surd$
   \rule{.48\textwidth}{0.5pt}
}
\makeatother

\begin{document}
\twocolumn[\maketitle]
\linespread{0.92}\selectfont

\section*{概要}
秘密画像共有(Secret Image Sharing; SIS)で分散保存した画像について,平文化せずに知覚ハッシュ pHash(perceptual hash)による類似画像検索を行う方式を提案する.
シェアのみを持つサーバでも検索できるよう,
\textbf{$k_1$ で pHash の符号だけを開示し,視覚的にはノイズ化した pHash 整合ダミー画像だけを見せる}
三段階開示モデルを設計した.ここで $1<k_1<k_2\le n$ とし,$k_1$ は検索のみ許可,$k_2$ は完全復元の閾値とする.
シェア数に応じて
\textbf{「ノイズ」「pHash 整合ダミー」「原画像」}
の三段階で情報を開示し,
検索精度と秘匿性の両立を図る.
MS COCO(Common Objects in Context)派生データによる評価では,
pHash 検索再現率・処理時間ともに平文と同等であることを確認した.

\section{はじめに}
医用画像や監視映像ではプライバシ侵害の可能性があるため,
画像を秘密共有した状態で検索したいという要求がある.
既存の類似画像検索(CBIR: Content-Based Image Retrieval)は画像や特徴量を平文で保持するため,
漏洩リスクが高い.
pHash は軽量で実用的な知覚特徴量である一方,
符号を平文で扱うと情報漏洩につながる.
また,既存の秘密画像共有(SIS)は
「復元する/しない」の二択であり,
\textbf{復元せずに検索だけを許可する中間状態}を扱えないという課題がある.
そこで,秘密共有した状態で検索を可能にする暗号化方式を提案する.

プライバシ保護 CBIR では,
CNN 特徴量を秘密分散や MPC により安全計算する方式が提案されている.
しかし,高次元特徴を前提とするため計算コストが高い.
加えて,平文特徴の類似度計算を前提に設計されており,
秘密共有や知覚暗号化された画像に対しては
検索精度が低下するか,そもそも適用対象外になりやすい.
段階的な情報開示も備えていない.
一方,pHash は低次元で高速だが,
暗号技術と統合した研究はほとんど存在しない.
特に,
\textbf{pHash 符号のみを保持したまま視覚情報を消去する設計}
は未検討である.

\section{提案法}
本研究で用いる pHash は,32$\times$32 画像の DCT 左上 8$\times$8 の符号から
64bit を得る軽量特徴であり,低周波構造のみを保持する.
二階層 Shamir 秘密分散は,同じシェア集合から
$k_1$ で検索専用の情報,$k_2$ で原画像を復元する段階閾値を与える.
本研究では,両者を組み合わせて三段階の情報開示を実現する.

\subsection{定式化(三段階 SIS)}
閾値は $1<k_1<k_2\le n$ とし,$k_1$ は検索権限,$k_2$ は原画像復元権限を表す.
配布した $n$ 枚のシェア集合を $A$,$r=|A|$ とし,秘密を $S_0,S_1,S_2$ で定義する.
\[
\text{output}(r)=
\begin{cases}
\text{noise} & (r < k_1) \\
\text{pHash 整合ダミー} & (k_1 \le r < k_2) \\
\text{原画像} & (r \ge k_2)
\end{cases}
\]
\noindent ここで $S_0$ は無意味なノイズ画像,$S_1$ は pHash 整合ダミー生成情報,$S_2$ は原画像復元情報を表す.
\noindent 安全性として $r<k_1$ で $I(S_1;A)=0$,かつ $r<k_2$ で $I(S_2;A)=0$ を要求する.
\noindent また $k_1 \le r < k_2$ では pHash による検索のみ可能($\mathrm{pHash}(S_1)=\mathrm{pHash}(I)$)とする.
\noindent 同じシェア集合から復元結果が段階的に変わる設計であり,
\textbf{秘密分散のシェア計算を3回行うわけではない.}
\begin{figure}[t]
\centering
\includegraphics[width=0.85\linewidth]{fig_scene_cropped.png}
\caption{三段階 SIS の概念図($k_1$:検索のみ,$k_2$:完全復元)}
\label{fig:three-stage}
\end{figure}

\subsection{pHash 整合ダミー}
低周波 DCT 符号のみを一致させ,
高周波成分をランダムノイズで置換することで,
pHash は一致するが視覚情報は失われた画像を生成する.
図\ref{fig:dummy-pipeline} に生成手順の要点を示す.
\begin{figure}[t]
\centering
\includegraphics[width=0.9\linewidth]{pipeline_top3.png}
\caption{pHash 整合ダミー生成の流れ(符号抽出→低周波のみの空間像→32$\times$32 空間)}
\label{fig:dummy-pipeline}
\end{figure}

\subsection{二階層 Shamir}
$n=5,\ k_1=2,\ k_2=4$ とし,
$k_1$ で検索権限,
$k_2$ で原画像復元を許可する.

\section{実験}
COCO val2017 公式配布\footnote{\url{http://cocodataset.org/#download}}から
seed=2025 で 500 枚を抽出し,
JPEG 劣化・ガンマ/輝度/コントラスト・回転・クロップ・ノイズ・透かし等の
20 種変換を適用して派生データを作成した.
各画像について,
(1) 32$\times$32 グレースケール化と pHash 計算,
(2) 低周波符号を固定した pHash 整合ダミー生成,
(3) $n=5,\ k_1=2,\ k_2=4$ の二階層 Shamir によるシェア化を行い,
平文 pHash と $k_1$ ダミー pHash の検索順位を比較した.

生成されたダミーは視覚的にはノイズだが pHash 符号は元画像と一致する.
同一画像のシェアから $k_1$ 復元した場合は Hamming 距離 0 近傍に集中し,
異なる画像のシェア同士では距離が大きく分離した.
一方 $k_1$ 未満の組合せでは距離がランダム同等(平均 20.8)となり,
pHash の漏洩が起きないことを確認した.
その結果,
\begin{itemize}
  \item オリジナルのみの1対1照合では Top-1 accuracy が 100\%(平文・ダミー一致)
  \item 全バリアントでは Recall@5=21.65\%,Recall@10=42.41\%(同一ID派生群)
  \item 検索時間は約 0.5 ms/query と平文と同等
\end{itemize}
であり,
pHash 整合ダミーが検索性能を劣化させないことを確認した.
図\ref{fig:resume-recall} と図\ref{fig:resume-time} に,オリジナルのみの再現率と時間の要約を示す.

\begin{figure}[t]
\centering
\includegraphics[width=0.95\linewidth]{recall_summary.png}
\caption{検索再現率(オリジナルのみ,plain vs dummy\_k1)}
\label{fig:resume-recall}
\end{figure}

\begin{figure}[t]
\centering
\includegraphics[width=0.95\linewidth]{time_summary.png}
\caption{検索時間(オリジナルのみ,plain vs dummy\_k1)}
\label{fig:resume-time}
\end{figure}


\section{おわりに}
pHash 符号一致ダミーと二階層秘密分散を用いた
段階的情報開示方式を提案した.
本方式により,
\textbf{画像を見せずに画像を検索する}という要求を
軽量かつ実用的に実現できることを示した.
今後は大規模データへの拡張や,
CNN 特徴とのハイブリッド化を検討する.

\begin{thebibliography}{9}
\bibitem{xia}
Z. Xia et al., ``A privacy-preserving CBIR scheme based on secret sharing,''
\textit{IEEE Access}, 2020.
\end{thebibliography}

\end{document}










